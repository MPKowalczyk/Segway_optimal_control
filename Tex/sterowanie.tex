\section{Sterowanie optymalne}
\label{sec:sterowanie_optymalne}

Podstawowym celem sterowania jest utrzymanie pojazdu w pozycji pionowej. Jest to niestabilny punkt równowagi. Sterowanie powinno uwzględniać ograniczenie maksymalnego wychylenia obiektu z tego położenia oraz maksymalne napięcie podawane na silnik prądu stałego. Oprócz tego wymaga się, żeby pojazd przemieścił się o zadaną odległość przy jak najmniejszym zużyciu energii. Wybrano więc następujący wskaźnik jakości sterowania:
\begin{equation}
Q=x^T(T)Ax(T)+\int\limits_{0}^{T}Ru(t)^2dt
\end{equation}
\noindent Gdzie:
\(x\) jest stanem obiektu.\newline
\(u\) jest wejściem