\section{Sterowanie optymalne}
\label{sec:sterowanie_optymalne}

Podstawowym celem sterowania jest utrzymanie pojazdu w pozycji pionowej. Jest to niestabilny punkt równowagi. Sterowanie powinno uwzględniać ograniczenie maksymalnego wychylenia obiektu z tego położenia oraz maksymalne napięcie podawane na silnik prądu stałego. Oprócz tego wymaga się, żeby pojazd przemieścił się o zadaną odległość przy jak najmniejszym zużyciu energii. Wybrano więc następujący wskaźnik jakości sterowania:
\begin{equation}
Q=x^T(T)Ax(T)+\int\limits_{0}^{T}Ru(t)^2dt
\end{equation}
\noindent Gdzie:\newline
\(x\) jest stanem obiektu.\newline
\(u\) jest wejściem obiektu (napięciem podawanym na silnik).\newline
\(T\) jest horyzontem czasowym.\newline
\(R\) jest współczynnikiem określającym wagę kwadratu sterowania.

\paragraph*{}
Należy wziąć pod uwagę również ograniczenia:
\begin{equation}
|u(t)|\leqslant u_{max}
\label{eq:u_max}
\end{equation}
\begin{equation}
|x_3(t)|\leqslant \phi_{max}
\label{eq:phi_max}
\end{equation}
\noindent Gdzie:\newline
\(u_{max}\) jest maksymalnym napięciem, które można podać na silnik.\newline
\(\phi_{max}\) jest maksymalnym wychyleniem pojazdu z niestabilnego punktu równowagi.

\paragraph*{}
Ograniczenie \eqref{eq:u_max} odnosi się do fizycznego ograniczenia napięciem akumulatora urządzenie. Nie jest możliwe podanie większego napięcia. Ograniczenie \eqref{eq:phi_max} ma na celu zabezpieczenie pojazdu przed upadkiem oraz zrzuceniem transportowanego obiektu.