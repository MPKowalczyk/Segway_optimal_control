\section{Sterowanie optymalne}
\label{sec:sterowanie_optymalne}

Podstawowym celem sterowania jest utrzymanie pojazdu w pozycji pionowej. Jest to niestabilny punkt równowagi. Sterowanie powinno uwzględniać ograniczenie maksymalnego wychylenia obiektu z tego położenia oraz maksymalne napięcie podawane na silnik prądu stałego. Oprócz tego wymaga się, żeby pojazd przemieścił się o zadaną odległość przy jak najmniejszym zużyciu energii. Wybrano więc następujący wskaźnik jakości sterowania:
\begin{equation}
Q=x^T(T)x(T)
\end{equation}
\noindent Gdzie:\newline
\(x\) jest stanem obiektu.\newline
\(T\) jest horyzontem czasowym.

\paragraph*{}
Należy wziąć pod uwagę również ograniczenia:
\begin{equation}
|u(t)|< u_{max}
\label{eq:u_max}
\end{equation}
\begin{equation}
|x_3(t)|< \phi_{max}
\label{eq:phi_max}
\end{equation}
\noindent Gdzie:\newline
\(u\) jest wejściem obiektu (napięciem podawanym na silnik).\newline
\(u_{max}\) jest maksymalnym napięciem, które można podać na silnik.\newline
\(\phi_{max}\) jest maksymalnym wychyleniem pojazdu z położenia pionowego.

\paragraph*{}
Ograniczenie \eqref{eq:u_max} odnosi się do fizycznego ograniczenia napięciem akumulatora urządzenia. Nie jest możliwe podanie większego napięcia. Ograniczenie \eqref{eq:phi_max} ma na celu zabezpieczenie pojazdu przed upadkiem oraz zrzuceniem transportowanego obiektu. Nierówność \eqref{eq:phi_max} została uwzględniona w zadaniu za pomocą funkcji kary. Zadanie więc zostało zmodyfikowane do następującej postaci:
\begin{equation}
\begin{aligned}
\dot x_1 &=x_2\\
\dot x_2 &=k_1x_2+k_2\frac{L}{M}\cos x_3+k_3x_4^2\sin x_3+k_4u\\
\dot x_3 &=x_4\\
\dot x_4 &=\frac{L}{M}\\
\dot x_5 &=f_5\\
Q &=x^T(T)x(T)
\end{aligned}
\label{eq:ss_penalty}
\end{equation}
\noindent Gdzie:
\begin{equation}
\begin{aligned}
L &=(k_5\cos x_3+k_6)u+(k_7\cos x_3+k_8)x_2+k_9\sin x_3+k_{10}x_4^2\sin x_3\cos x_3\\
M &=k_{11}+k_{12}\cos ^2x_3\\
f_5 &=
	\begin{cases}
	\frac{K(x_3-\phi_{max})^2}{2}, & \text{kiedy } \phi_{max}\leqslant x_3\\
	0, & \text{kiedy } -\phi_{max}<x_3<\phi_{max}\\
	\frac{K(\phi_{max}+x_3)^2}{2}, & \text{kiedy } x_3\leqslant -\phi_{max}
	\end{cases}
\end{aligned}
\end{equation}
Dla zadania opisanego równaniami \eqref{eq:ss_penalty} wyznaczono równania sprzężone:
\begin{equation}
\dot \psi=-\frac{\partial f}{\partial x}\psi
\end{equation}
\noindent Gdzie:\newline
\(\psi\) jest wektorem funkcji sprzężonych \([\psi_1, \psi_2, \psi_3, \psi_4, \psi_5]^T\).\newline
\(f\) jest prawymi stronami układu równań stanu \eqref{eq:ss_penalty}.
\begin{equation}
\frac{\partial f}{\partial x}=\begin{bmatrix}
0 & 0 & 0 & 0 & 0\\
1 & k_2\cos(x_3)\frac{\partial S}{\partial x_2}+k_1 & 0 & \frac{\partial S}{\partial x_2} & 0\\
0 & k_3x_4^2\cos x_3+k_2\frac{\partial S}{\partial x_3}\cos x_3-k_2S\sin x_3 & 0 & \frac{\partial S}{\partial x_3} & \frac{f_5}{x_3}\\
0 & 2k_3x_4\sin x_3+k_2\frac{\partial S}{x_4}\cos x_3 & 1 & \frac{\partial S}{\partial x_4} & 0\\
0 & 0 & 0 & 0 & 0
\end{bmatrix}
\end{equation}
\noindent Gdzie:\newline
\(S=\frac{L}{M}\) jest funkcją wprowadzoną w celu uproszczenia obliczeń.
\begin{equation}
\frac{\partial f_5}{\partial x_3}=
	\begin{cases}
	K(x_3-\phi_{max}), & \text{kiedy } \phi_{max}\leqslant x_3\\
	0, & \text{kiedy } -\phi_{max}<x_3<\phi_{max}\\
	K(x_3+\phi_{max}), & \text{kiedy } x_3\leqslant -\phi_{max}
	\end{cases}
\end{equation}
Warunek końcowy równań sprzężonych jest następujący:
\begin{equation}
\psi(T)=-\frac{\partial Q}{\partial x(T)}
\end{equation}
Wiadomo, że równania sprzężone spełniają następujące równanie:
\begin{equation}
\frac{\partial Q(u,x(0))}{\partial x(0)}=-\psi(0)
\label{eq:check_psi}
\end{equation}
Postanowiono wykorzystać tę równość do sprawdzenia poprawności wyznaczonych równań sprzężonych. W tym celu konieczne było rozwiązanie równań sprzężonych do tyłu w celu wyznaczenia wartości \(-\psi(0)\). Pochodna \(\frac{\partial Q(u,x(0))}{\partial x(0)}\) została przybliżona za pomocą ilorazów różnicowych:
\begin{equation}
\frac{\partial Q(u,x(0))}{\partial x_k(0)}\approx\frac{Q(u,x_1(0),\dots, x_k(0)+\epsilon,\dots,x_n(0))}{\epsilon}
\end{equation}
\noindent Gdzie:\newline
\(\epsilon\) jest bardzo małym przyrostem (przyjmowano \(\epsilon=10^{-7}\)).
\paragraph*{}
Dla różnych przypadków testowych (różne warunki początkowe i ograniczenia) sprawdzano błąd równości \eqref{eq:check_psi}. Przykładowy błąd dla prostego zadania testowego wynosi:
\begin{equation}
\frac{\partial Q(u,x(0)}{\partial x(0)}+\psi(0)=
\begin{bmatrix}
5.14\cdot 10^{-8}\\
2.27\cdot 10^{-7}\\
1.87\cdot 10^{-5}\\
2.96\cdot 10^{-6}\\
-1.63\cdot 10^{-9}
\end{bmatrix}
\end{equation}
Błędy są bardzo małe. Na podstawie powyższego wyniku stwierdzono, że wyznaczone równania sprzężone są poprawne.