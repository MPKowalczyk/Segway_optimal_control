\section{Postawienie problemu sterowania optymalnego}
\label{sec:sterowanie_optymalne}

Podstawowym celem sterowania jest utrzymanie pojazdu w pozycji pionowej. Jest to niestabilny punkt równowagi. Sterowanie powinno uwzględniać ograniczenie maksymalnego wychylenia obiektu z tego położenia oraz maksymalne napięcie podawane na silnik prądu stałego. Oprócz tego wymaga się, żeby pojazd przemieścił się o zadaną odległość przy jak najmniejszym zużyciu energii. Wybrano więc następujący wskaźnik jakości sterowania:
\begin{equation}
Q=\frac{1}{2}x(T)^\text{T}x(T)
\end{equation}
\noindent gdzie:\newline
\(x\) jest stanem obiektu\newline
\(T\) jest horyzontem czasowym.

\paragraph*{}
Należy wziąć pod uwagę również ograniczenia:
\begin{equation}
|u(t)|\leqslant u_{max}
\label{eq:u_max}
\end{equation}
\begin{equation}
|x_3(t)|\leqslant \phi_{max}
\label{eq:phi_max}
\end{equation}
\noindent gdzie:\newline
\(u\) jest sterowaniem (napięciem podawanym na silnik)\newline
\(u_{max}\) jest maksymalnym napięciem, które można podać na silnik\newline
\(\phi_{max}\) jest maksymalnym wychyleniem pojazdu z położenia pionowego.

\paragraph*{}
Ograniczenie \eqref{eq:u_max} odnosi się do fizycznego ograniczenia napięcia akumulatora. Nie jest możliwe podanie większego napięcia. Ograniczenie \eqref{eq:phi_max} ma na celu zabezpieczenie pojazdu przed upadkiem oraz zrzuceniem transportowanego obiektu. Nierówność \eqref{eq:phi_max} została uwzględniona w zadaniu za pomocą funkcji kary. Zadanie więc zostało zmodyfikowane do następującej postaci:
\begin{equation}
\begin{aligned}
\dot x_1 &=x_2\\
\dot x_2 &=k_1x_2+k_2\frac{L}{M}\cos x_3+k_3x_4^2\sin x_3+k_4u\\
\dot x_3 &=x_4\\
\dot x_4 &=\frac{L}{M}\\
\dot x_5 &=f_5\\
Q &=\frac{1}{2}\sum\limits_{k=1}^4 x_k(T)^2+x_5(T)
\end{aligned}
\label{eq:ss_penalty}
\end{equation}
\noindent gdzie:\newline
\(L\) i \(M\) są oznaczeniami wprowadzonymi w równaniu \eqref{eq:nonlinear_ss}.
\begin{equation}
\begin{aligned}
f_5 &=
	\begin{cases}
	\frac{K(x_3-\phi_{max})^2}{2}, & \text{kiedy } \phi_{max}\leqslant x_3\\
	0, & \text{kiedy } -\phi_{max}<x_3<\phi_{max}\\
	\frac{K(\phi_{max}+x_3)^2}{2}, & \text{kiedy } x_3\leqslant -\phi_{max}
	\end{cases}
\end{aligned}
\end{equation}
Dla zadania opisanego równaniami \eqref{eq:ss_penalty} wyznaczono funkcję sprzężoną:
\begin{equation}
\dot \psi=-\frac{\partial f}{\partial x}\psi
\end{equation}
\noindent gdzie:\newline
\(\psi\) jest wektorem funkcji sprzężonych \([\psi_1, \psi_2, \psi_3, \psi_4, \psi_5]^T\)\newline
\(f\) jest prawymi stronami układu równań stanu \eqref{eq:ss_penalty}.
\begin{equation}
\frac{\partial f}{\partial x}=\begin{bmatrix}
0 & 0 & 0 & 0 & 0\\
1 & k_2\cos(x_3)\frac{\partial S}{\partial x_2}+k_1 & 0 & \frac{\partial S}{\partial x_2} & 0\\
0 & k_3x_4^2\cos x_3+k_2\frac{\partial S}{\partial x_3}\cos x_3-k_2S\sin x_3 & 0 & \frac{\partial S}{\partial x_3} & \frac{\partial f_5}{\partial x_3}\\
0 & 2k_3x_4\sin x_3+k_2\frac{\partial S}{x_4}\cos x_3 & 1 & \frac{\partial S}{\partial x_4} & 0\\
0 & 0 & 0 & 0 & 0
\end{bmatrix}
\end{equation}
\noindent gdzie:\newline
\(S=\frac{L}{M}\) jest funkcją wprowadzoną w celu uproszczenia obliczeń. Pochodne funkcji \(S\) można znaleźć w skryptach dołączonych do pracy.
\begin{equation}
\frac{\partial f_5}{\partial x_3}=
	\begin{cases}
	K(x_3-\phi_{max}), & \text{kiedy } \phi_{max}\leqslant x_3\\
	0, & \text{kiedy } -\phi_{max}<x_3<\phi_{max}\\
	K(x_3+\phi_{max}), & \text{kiedy } x_3\leqslant -\phi_{max}
	\end{cases}
\end{equation}
Warunek końcowy funkcji sprzężonej jest następujący:
\begin{equation}
\psi(T)=-\frac{\partial Q}{\partial x(T)}=
\begin{bmatrix}
-x_1(T)\\
-x_2(T)\\
-x_3(T)\\
-x_4(T)\\
-1
\end{bmatrix}
\end{equation}
Wiadomo, że funkcja sprzężona spełnia następujące równanie:
\begin{equation}
\frac{\partial Q(u,x(0))}{\partial x(0)}=-\psi(0)
\label{eq:check_psi}
\end{equation}
Postanowiono wykorzystać tę równość do sprawdzenia poprawności wyznaczonej funkcji sprzężonej. W tym celu konieczne było rozwiązanie równań sprzężonych do tyłu w celu wyznaczenia wartości \(-\psi(0)\). Pochodna \(\frac{\partial Q(u,x(0))}{\partial x(0)}\) została przybliżona za pomocą ilorazów różnicowych:
\begin{equation}
\frac{\partial Q(u,x(0))}{\partial x_k(0)}\approx\frac{Q(u,x_1(0),\dots, x_k(0)+\epsilon,\dots,x_n(0))}{\epsilon}
\end{equation}
\noindent gdzie:\newline
\(\epsilon\) jest bardzo małym przyrostem (przyjmowano \(\epsilon=10^{-7}\)).
\paragraph*{}
Dla różnych przypadków testowych (różne warunki początkowe i ograniczenia) sprawdzano błąd równości \eqref{eq:check_psi}. Przykładowe wartości dla prostego zadania testowego wynoszą:
\begin{equation}
\frac{\partial Q(u,x(0))}{\partial x(0)}=
\begin{bmatrix}
\begin{aligned}
-&9.67666569\cdot 10^{-2}\\
-&2.85626527\cdot 10^{0}\\
&3.40155712\cdot 10^{1}\\
&1.35501054\cdot 10^{1}\\
&1.00000000\cdot 10^{0}
\end{aligned}
\end{bmatrix}
\qquad
-\psi(0)=
\begin{bmatrix}
\begin{aligned}
-&9.67667099\cdot 10^{-2}\\
-&2.85626538\cdot 10^{0}\\
&3.40155622\cdot 10^{1}\\
&1.35501039\cdot 10^{1}\\
&1.00000000\cdot 10^{0}
\end{aligned}
\end{bmatrix}
\end{equation}
Błędy są bardzo małe. Na podstawie powyższego wyniku stwierdzono, że wyznaczone równania sprzężone są poprawne.\\
W trakcie numerycznego rozwiązywania wstecz równań sprzężonych wyliczano również gradient \eqref{eq:Grad}.
\begin{equation}
\begin{aligned}
&z(t_i) = \frac{\partial Q}{\partial u_i}=-\int\limits_{t_i}^{t_{i+1}}\frac{\partial H}{\partial u}dt&\\
&\dot{z}(t) = \frac{\partial H}{\partial u_i}, t\in [t_i,t_{i+1}]&\\
&z(t_{i+1}) = 0
\label{eq:Grad}
\end{aligned}
\end{equation}
Wymiar wektora \textit{z} jest zależny od wyboru liczby węzłów strukturalnych. Sprawdzenia poprawności wyliczonych gradientów dokonano poprzez porównanie otrzymanych wyników z ilorazem różnicowym \eqref{eq:checkGrad}.
\begin{equation}
\label{eq:checkGrad}
\frac{Q(u+\Delta u)-Q(u)}{\Delta u}
\end{equation}
Wyniki otrzymane dla prostego zadania testowego:
\begin{equation}
\frac{\partial Q}{\partial u}=
\begin{bmatrix}
1.64610454\cdot 10^{-1}\\
1.27314436\cdot 10^{-1}\\
1.02532044\cdot 10^{-1}\\
8.50547635\cdot 10^{-2}\\
7.28772423\cdot 10^{-2}\\
6.68479648\cdot 10^{-2}\\
6.71731968\cdot 10^{-2}\\
7.75069815\cdot 10^{-2}\\
9.76715801\cdot 10^{-2}
\end{bmatrix}
\qquad
\frac{Q(u+\Delta u)-Q(u)}{\Delta u}=
\begin{bmatrix}
1.64610574\cdot 10^{-1}\\
1.28604931\cdot 10^{-1}\\
1.03553459\cdot 10^{-1}\\
8.58912052\cdot 10^{-2}\\
7.35804306\cdot 10^{-2}\\
6.74685329\cdot 10^{-2}\\
6.77647360\cdot 10^{-2}\\
7.81427855\cdot 10^{-2}\\
9.84478631\cdot 10^{-2}\\
\end{bmatrix}
\end{equation}
Błędy są bardzo małe. Na podstawie powyższego wyniku stwierdzono, że wyznaczony gradient jest poprawny.