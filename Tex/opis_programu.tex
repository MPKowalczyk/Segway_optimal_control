\section{Opis używanego programu}
\label{sec:opis_uzywanego_programu}

Do wyznaczenia sterowania optymalnego wykorzystano oprogramowania \textit{Matlab}. W tym celu napisano główny skrypt, który wywoływał konieczne do wykonania zadania. W pierwszej części skryptu ustawiane są parametry symulacji, takie jak czas trwania i częstotliwość próbkowania.
\paragraph*{}
W drugiej części skryptu podawane są parametry symulowanego obiektu, takie jak:
\begin{itemize}
\item masa Segway'a
\item moment bezwładności
\item stała elektromotoryczna silnika
\item stała momentu silnika
\item maksymalne napięcie podawane na silnik
\end{itemize}
Następnie na podstawie tych parametrów wyliczone zostają współczynniki \(k_1,\dots,k_{12}\) modelu matematycznego obiektu \eqref{eq:nonlinear_ss}.
\paragraph*{}
Kolejna część programu wykorzystuje algorytm RK4 do rozwiązania numerycznego równań różniczkowych modelu. Następnie wykorzystano ten sam algorytm do rozwiązania równań sprzężonych w tył. Poprawność równań sprzężonych jest również w tej części skryptu sprawdzona.
\paragraph*{}
W ostatnim etapie skryptu uruchamiany jest algorytm optymalizacji BFGS (Broyden-Fletcher-Goldfarb-Shanno). Wyznacza on sterowanie, które minimalizuje wybrany wskaźnik jakości. Funkcja realizująca algorytm BFGS wywołuje funkcję poszukiwania na kierunku, która składa się z metody kontrakcji i ekspansji. Przebieg algorytmu optymalizacji (sterowanie w kolejnych iteracjach) jest zapisywany. Dla znalezionego sterowania optymalnego wyznaczany jest przebieg zmiennych stanu obiektu. Przebiegi zmiennych stanu dla wyznaczonego sterowania oraz samo sterowania zostają następnie pokazane na wykresach.