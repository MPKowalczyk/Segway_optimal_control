\section{Wnioski}
\label{sec:wnioski}

Podczas realizacji niniejszej pracy napotkano na szereg trudności. Punktem przełomowym wydaje się być poprawne rozwiązanie równań sprzężonych. Bardzo pomocny na tym etapie okazał się przybornik \textit{Symbolic Math Toolbox} (pakiet \textit{MATLAB}).

Dużo do myślenia pozostawiła implementacja użytych metod numerycznych. Skracając długość kroku algorytmu (RK4) otrzymane rezultaty coraz bardziej zbiegały do wzorcowych, zwróconych przez wbudowany solver, \textit{ode45}. W pewnym momencie różnica ta przestała maleć. Co więcej, czas wykonania programu wzrastał niewspółmiernie do otrzymanej poprawy. Dla dynamiki Segwaya wystarczający krok to 0.001 (\ref{fig:wychylenie_1khz}).

Bardzo istotnym jeśli chodzi o czas wykonania programu okazał się być dobór początkowego kroku poszukiwania na kierunku oraz współczynników  ekspansji i kontrakcji. Wytłumaczeniem tego zjawiska może być ogromna ilość wywołań przywołanych procedur w trakcie całego algorytmu wyliczającego sterowanie optymalne. Dla większości eksperymentów korzystano z parametrów: $StepZeroLength = 1$, $WspEksp = 3$, $WspKontr = 0.5$. Należy zwrócić uwagę na wielkości jakie przyjmuje zmienna decyzyjna, a także uważnie obserwować wpływ dokonywanych zmian.

Przeprowadzając dużą liczbę różnych eksperymentów zauważono wpływ wybory sterowania początkowego na efektywność działania algorytmu. Wygodnym pomysłem, który zapobiegał przed "utknięciem" programu, było losowanie rozwiązania początkowego w każdym przedziale strukturalnym z osobna.



